\documentclass[letterpaper,11pt]{article}

\usepackage{fullpage}
\usepackage{anysize}
\marginsize{2.5cm}{2cm}{1cm}{2cm} %iz, der, arriba, abajo

\usepackage{multicol} % This is so we can have multiple columns of text side-by-side
\columnsep=20pt % This is the amount of white space between the columns in the poster
\columnseprule=0pt % This is the thickness of the black line between the columns in the poster

\usepackage[svgnames]{xcolor} % Specify colors by their 'svgnames', for a full list of all colors available see here: http://www.latextemplates.com/svgnames-colors

\usepackage{times} % Use the times font
%\usepackage{palatino} % Uncomment to use the Palatino font

\usepackage{graphicx} % Required for including images
\graphicspath{{figures/}} % Location of the graphics files
\graphicspath{ {images/} }
\usepackage{float}
\usepackage{mathdots} % para el comando \iddots
\usepackage{mathrsfs} % para formato de letra
\usepackage{amssymb, amsmath, amsbsy} % simbolitos
\usepackage{booktabs} % Top and bottom rules for table
\usepackage[font=small,labelfont=bf]{caption} % Required for specifying captions to tables and figures
\usepackage{amsfonts, amsmath, amsthm, amssymb} % For math fonts, symbols and environments
\usepackage{wrapfig} % Allows wrapping text around tables and figures
\usepackage{listings}
\usepackage{matlab-prettifier}
\usepackage[latin1]{inputenc}
\usepackage[spanish]{babel} %para agregar tildes 
\usepackage{mathrsfs}
\usepackage{multicol,caption}
\usepackage{courier}

%%%%%%%%%%%%%%%%%%%%%%%%%%%%%%%%%%%%%%%%%%%%% Pie de pagina
\usepackage{fancyhdr} % Cabeceras/Pies
\pagestyle{fancy} % Cabeceras/Pies
\usepackage{lastpage}
\addtolength{\headwidth}{\marginparsep}
% Para el resto de p ́aginas
\rhead{} 
\renewcommand{\headrulewidth}{0.0pt} 
\lfoot{foot page \LaTeX{}}

\rfoot{\thepage/\pageref{LastPage}} 
\renewcommand{\footrulewidth}{2pt}
%%%%%%%%%%%%%%%%%%%%%%%%%%%%%%%%%%%%%%%%%%%%%%%%%%%%%%%%%%%

\begin{document}

%\begin{picture}(10,10)(-312,65)
%    \includegraphics[width=6cm]{figures/logo_uii.png} \\
%\end{picture}

\tiny \texttt{}\\


\begin{center}
\LARGE \color{Black} {Title} \\ 
\end{center}
\color{Black}
\tiny \texttt{}
%%%%%%%%%%%%%%%%%%%%%%%%%%%%%%%%%%%%%%%%%%%%%%%%%%%
%%%%%%%%%%%%%%%%%%%%%%%%%%%%             ENCABEZADO
%%%%%%%%%%%%%%%%%%%%%%%%%%%%%%%%%%%%%%%%%%%%%%%%%%%
\large \textbf{}  \\

\begin{center}
Author name\\
\small email@direction.co\\
\small University \\
\small City\\
\end{center}


\vspace{0.2cm}



\vspace{0.15cm} % A bit of extra whitespace 

\renewcommand{\refname}{}


%----------------------------------------------------------------------------------------
\begin{multicols*}{2} % This is how many columns your poster will be broken into, a portrait poster is generally split into 2 columns

\section{first section}

\begin{figure}[H]
\begin{center}
\includegraphics[scale=0.5]{figures/ejemplo_img.png}
\caption{example image} 
\end{center}
\end{figure}

\subsection*{Nuevo formato de encabezado}



\subsection*{Example Item}

\begin{itemize}
\item item1

\item item2

\end{itemize}

\section*{References}

\begin{thebibliography}{a}
\bibliographystyle{ieee}

\bibitem @online{,
title = {title},
url = {https://www.url.com/},
}

\bibitem @online{,
author = {Author name},
title = {title},
year = {2017},
url = {https://www.url.com},
}


\end{thebibliography}



\end{multicols*}
\end{document}




%------------------------------------------------




