\markboth{INTRODUCCIÓN}{}
\thispagestyle{fancyplain}
\chapter*{Introducción}


En la introducción, el autor presenta y señala la importancia, el origen (los antecedentes teóricos y prácticos), los objetivos, los alcances, las limitaciones, la metodología empleada, el significado que el estudio tiene en el avance del campo respectivo y su aplicación en el área investigada. No debe confundirse con el resumen y se recomienda que la introducción tenga una extensión de mínimo 2 páginas y máximo de 4 páginas.\\

La presente plantilla maneja la fuente Arial para el texto de los párrafos y para los títulos y subtítulos. Sin embargo, es posible sugerir otras fuentes tales como Garomond, Calibri, Cambria o Times New Roman, que por claridad y forma, son adecuadas para la edición de textos académicos.\\ 

Esta sección se encabeza con la palabra introducción, escrita con minúscula (en la primera línea), con un espaciado anterior de 30 puntos y posterior de 10 puntos, interlineado sencillo y en letra negrilla de 20 puntos (en este caso se usa Arial).\\

La presente plantilla tiene en cuenta los aspectos generales recomendados por la Norma Técnica Colombiana - NTC 1486, con el fin que sea usada para la presentación final de los trabajos finales de grado de los programas académicos (en cualquiera de sus modalidades o sus niveles de formación) de la Facultad de Ingeniería, desarrollados en la Universidad de Ibagué en Colombia.\\

Las márgenes deben ser de 2,54 centímetros (1 pulgada) en la parte superior, inferior y exterior y de 3,6 centímetros en la margen interna (a partir de márgenes simétricos). La plantilla está diseñada para imprimir por lado y lado en hojas tamaño carta, y está construida por secciones (Capítulo 1, Capítulo 2, Capitulo 3, Capítulo 4, Capitulo 5,). El encabezado de las páginas impares contendrá el título del trabajo en letra Arial 8 puntos, el encabezado en las páginas pares contendrá la sección que se desarrolla. En el pie de página se ubica además del número de página, el nombre de los autores en páginas impares y el nombre del programa -tipo de trabajo- año de presentación, en las páginas pares (en letra Arial de 8 puntos, de acuerdo al formato presentado en esta plantilla). \\

El título de cada capítulo debe estar numerado y comenzar en una hoja independiente (página impar) y con el mismo formato del título Introducción (escrita con minúscula, en la primera línea, con un espaciado anterior de 30 puntos y posterior de 10 puntos e interlineado 1.15 y en letra de 16 puntos y negrilla. El texto debe llegar hasta la margen inferior establecida. Se debe evitar títulos o subtítulos solos al final de la página o renglones sueltos. \\

Si se requiere ampliar la información sobre normas adicionales para la escritura se puede consultar la norma NTC 1486 en la Base de datos del ICONTEC (Normas Técnicas Colombianas) disponible en el portal del SINAB de la Universidad Nacional de Colombia www.sinab.unal.edu.co, en la sección “Recursos bibliográficos” opción “Bases de datos”.  Este portal también brinda la posibilidad de acceder a un instructivo para la utilización de Microsoft Word y Acrobat Professional.\\

La tesis o trabajo de investigación se debe escribir con interlineado 1.15 y después de punto aparte a una interlínea (una veces la tecla Enter). La redacción debe ser impersonal y genérica. La numeración de las hojas sugiere que las páginas preliminares se realicen en números romanos en mayúscula y las demás en números arábigos, en forma consecutiva a partir de la introducción que comenzará con el número 1. La cubierta y la portada no se numeran pero si se cuentan como páginas.\\

El tamaño de letra sugerido y teniendo en cuenta la familia fuente Arial de 11 puntos para el texto de estilo “Párrafo”, Arial para los títulos, de 16 puntos (estilo “Título Primer nivel”) y de 14 y 12 para los subtítulos (estilos “Título segundo nivel” y “Título tercer nivel”, respectivamente). \\

La extensión máxima de un documento de trabajo de grado, se sugiere en no más de 50 páginas en programas de pregrado y en no más de 60 páginas en programas de posgrado, sin considerar páginas de anexos ni preliminares.\\

No se debe utilizar una numeración adicional a la que es propia de la tabla de contenido. Tampoco utilice numeración compuesta como 13A, 14B ó 17 bis.  Si requiere hacer enumeración dentro de apartados, utilice viñetas ( $\blacksquare$ ) o números romanos en minúscula (i, ii, etc). Los Anexos se indican con letras alfabéticas en mayúscula, según su orden de llamado en el texto (Anexo A, etc.). \\

Para resaltar, puede usarse letra cursiva o negrilla. Los términos de otras lenguas que aparezcan dentro del texto se deben escribir en cursiva.\\
 
