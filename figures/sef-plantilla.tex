%======NO EDITAR ====================
\documentclass[12pt,spanish,letter]{article}

\usepackage[latin1]{inputenc}% Tildes
%\usepackage[spanish,english]{babel}
\usepackage[spanish]{babel}
\usepackage{graphicx}
\usepackage{amsmath, amsthm, amsfonts}
%\usepackage[hang,small,bf]{caption} %hang indenta la leyenda
\usepackage[small,bf]{caption}
\setlength{\captionmargin}{20pt}
\usepackage{indentfirst} %para indentar el primer parrafo de la seccion


% Normas APA utiliza el mismo tama�o
% de letra fuente para secciones, 
% subsecciones y subsubsecciones
\usepackage{titlesec}
\titleformat*{\section}{\large\bfseries}
\titleformat*{\subsection}{\normalfont\bfseries}
\titleformat*{\subsubsection}{\normalfont\bfseries}
\titleformat*{\paragraph}{\normalfont\bfseries}
\titleformat*{\subparagraph}{\normalfont\bfseries}

\usepackage{authblk}
\renewcommand\Authand{  }
\renewcommand\Authands{,  }

\usepackage{hyperref}
\usepackage[none]{hyphenat}% Sin guiones al final de oraciones
\tolerance=6000
\usepackage{apacite}%Stilo citaciones APA (debe ir despues de "hyperref "

\usepackage{geometry}
\geometry{right=1in, left=1in}
\geometry{top=1in, bottom=1in}
%\geometry{includehead}
\geometry{includefoot}
\textheight = 8.5in

\usepackage{fancyhdr}%
\pagestyle{fancy}
\usepackage{lastpage}

\fancypagestyle{plain}% Para la primera p�gina
{%
%  \setlength{\hoffset}{1.0in}
  \setlength{\topmargin}{-0.5in}
  \setlength{\headheight}{1.05in}
  \setlength{\headsep}{-0.0in}
  \setlength{\headwidth}{\textwidth}
  \setlength{\footskip}{2.0in}
  \fancyhead[l]{Memorias\\    
    {\bf XV Semana de la Ense�anza de la F�sica}\\
    SEF {\bf Vol. 1}: \thepage--\pageref{LastPage} (2012).
  }
  \fancyhead[c]{\hspace{10.4cm}
    \includegraphics[height=1.0cm, width=2.0cm]{logo}\\
  }
  \fancyhead[r]{
    \includegraphics[height=2.0cm, width=2.0cm]{logoUD_v1}
  }
  
}
%====================================

\pagestyle{myheadings}
%=====Coloque aqui el titulo o cabecera de pagina (running head) (una parte significativa del titulo)
% Maximo 50 caracteres
\markright{\normalfont PLANTILLA DE LA SEMANA DE LA ENSENANZA}

%====== Editar titulo autotres ======
\title{\Large \bf
Plantilla Semana de la Ense�anza de la F�sica}
\author[1]{Autor1}
\author[2]{Autor2}
\author[3]{Autor3}
\affil[1]{Afiliacion1}
\affil[2,3]{Afiliacion2}
\date{\normalsize Recibido: xxxx Aceptado: xxxx Publicado: xxxx\\
Todos los derechos reservados-SEF \copyright{} 2012}


%====================================

%===================================%
%===================================%
%       INICIO DOCUMENTO            %
%===================================%
%===================================%

\begin{document}

%======NO EDITAR ====================
\thispagestyle{fancy}
\maketitle
\thispagestyle{plain}
\let\oldthefootnote\thefootnote
\renewcommand{\thefootnote}{\fnsymbol{footnote}}
%====================================

%====EDITAR==========================
%correos electronicos de los autores.
\footnotetext[0]{\url{correo@autor1}}
\footnotetext[0]{\url{correo@autor2}}
\footnotetext[0]{\url{correo@autor3}}
%====================================

%======NO EDITAR ====================
\let\thefootnote\oldthefootnote
\setlength{\headheight}{0.65in}
\setlength{\textheight}{8.60in}
\pagestyle{myheadings}
%====================================

{\bf Abstract.} Abstract, keywords, resumen and descriptores can not exceed the first page. Abstract, keywords, resumen and descriptores can not exceed the first page. Abstract, keywords, resumen and descriptores can not exceed the first page. Abstract, keywords, resumen and descriptores can not exceed the first page. \\
{\bf{Keywords:}} Maxwell-Boltzmann Distribution, Ideal gas kinetic theory, physics education\\.

%======NO EDITAR ====================
%\renewcommand{\tablename}{{\bf Tabla}} %"Cuadro" por "Tabla"
\renewcommand{\tablename}{Tabla} %"Cuadro" por "Tabla"
%\renewcommand{\figurename}{{\bf \emph{Figura}}} %"Figure" por "Figura"
%\setlength{\parindent}{2em}
%====================================

{\bf Resumen.} El Abstract, keywords, resumen y descriptores no pueden exceder la primera pagina del articulo. El Abstract, keywords, resumen y descriptores no pueden exceder la primera pagina del articulo. El Abstract, keywords, resumen y descriptores no pueden exceder la primera pagina del articulo.\\
{\bf{Descriptores:}} Distribuci�n de Maxwell-Boltzmann, teor�a cin�tica de gases, ense�anza de la f�sica.\\

\section{ Introducci�n}

El titulo del articulo va en negrillas, 16pt, centrado. El nombre de los autores va en tama�o 14pt, centrado. Los t�tulos de las secciones y subsecciones van justificados a la derecha, numerados, en negrilla y en tama�o 14pt y 12 pt respectivamente. 

\subsection{Subseccion}

Si dentro de una subseccion hay otro nivel de titulo, este debe ir indentado, en negrilla y a 12pt. El texto subsecuente debe ir enseguida de este nivel de titulo. 

{\bf Otro nivel de titulo.} Y en seguida el texto.Y en seguida el texto.Y en seguida el texto.Y en seguida el texto.Y en seguida el texto.Y en seguida el texto.Y en seguida el texto.Y en seguida el texto.Y en seguida el texto.

{\bf Otro nivel de titulo.} Y en seguida el texto.Y en seguida el texto.Y en seguida el texto.Y en seguida el texto.Y en seguida el texto.Y en seguida el texto.Y en seguida el texto.Y en seguida el texto.Y en seguida el texto.

\section{Tablas, figuras, ecuaciones y cantidades fisicas}

Las tablas, figuras y ecuaciones se citan en el texto bajo normas APA. Las leyendas de tablas se ubican en la parte de arriba de las mismas. Para las graficas, en la parte de abajo. En ambos casos las leyendas deben ir centradas y a una indentaci�n de distancia a las m�rgenes derecha e izquierda (ver la leyenda de la figura 1). 
\begin{table}[h]
  \centering
  \caption{\footnotesize Ejemplo tabla.}% Este ejemplo fue tomado de la plantilla para manuscritos APA 5ta edici�n. Tama�o de la leyenda 11pt }
  \label{tab:tab1}
  \begin{tabular}{lcc}\hline
    & \multicolumn{2}{c}{Factor 2} \\ \cline{2-3}
    Factor 1  & Condition A  & Condition B   \\ \hline
    First     & 586 (231)    & 649 (255)     \\
    &    2.2       &    7.5        \\
    Second    & 590 (195)    & 623 (231)     \\
    &    2.8       &    2.5        \\ \hline
  \end{tabular}
\end{table}

Igualmente, se ilustra un ejemplo de una figura, utilizando el logo de la Semana de la ense�anza de la f�sica.

\begin{figure}[h]
  \centering
  \includegraphics[width=8.0cm, height=6.0cm]{logo}
  \caption{\footnotesize Logo SEF. Logo institucionalizado para la semana de la ense�anza de la f�sica.  Tama�o de la leyenda 11pt }
  \label{fig:fig1}
\end{figure}

Note que la palabra ``Tabla \#'' y la palabra ``Figura \#'' van en negrillas. Por ultimo, se ilustra un ejemplo para las ecuaciones:

\begin{equation}
I = \! \int_{-\infty}^\infty f(x)\,dx \label{eq:fine}.
\end{equation}

las cuales deben ir numeradas a lo largo del texto para su f�cil referencia.

\subsection{Cantidades f�sicas}
Se debe utilizar el Sistema Internacional de Unidades

\section{Citaci�n bajo normas APA}

Al igual que las figuras, tablas y ecuaciones, las referencias a citas bibliogr�ficas se realizan bajo normas APA.

Este es un ejemplo. El modelo de radiaci�n ilustrado \cite{salamanca} fue ilustrado en Colombia por \citeA{reyes,salamanca}, pero tambi�n fue utilizado en otras latitudes de la regi�n colombiana \cite{salamanca, reyes}.\\

\subsection{Margenes, parrafos, etc}
Las margenes son de una pulgada por cada lado. El interlineado debe ser sencillo para los p�rrafos y el espacio entre parrafos. La indentacion


\section{Compilaci�n en \LaTeX}

Para las personas que no deseen realizar su manuscrito en windows, se comentan algunas cuestiones t�cnicas respecto a la compilaci�n.

El c�digo fuente del articulo en \LaTeX se debe puede compilar con \LaTeX~ o Pdf\LaTeX~ siempre y cuando, las gr�ficas est�n en formato \texttt{eps} (para el primer caso) o en \texttt{png}, \texttt{gif} y \texttt{jpg} (para el segundo). Esta plantilla cuenta con un ejemplo de archivo para crear las referencias dentro del texto utilizando normas APA (\texttt{ejemplo-bib-SEF-v3.bib}). Primero se debe compilar el codigo \texttt{.tex}, luego el codigo \texttt{.aux} generado por la enterior compilacion y compilar de nuevo, dos veces, el codigo \texttt{.tex}. 

\section{Conclusiones}

El articulo debe contener una seccion de conclusiones. Se espera que esta seccion sea para conlcuir y discutir sobre los resultados del trabajo. De otro lado, se espera que tambien sea un espacio para recomendaciones y propuestas futuras de trabajo.\\

{\bf Agradecimientos.} Se utiliza igual que en ``otros niveles de titulo'' y es para resaltar las contribuciones que hicieron posible el articulo.

\bibliographystyle{apacite}
\bibliography{ejemplo-bib-SEF-v3}


\end{document}